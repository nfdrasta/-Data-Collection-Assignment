\documentclass[options]{article}

 \usepackage[
    top    = 2.75cm,
    bottom = 2.50cm,
    left   = 4.00cm,
    right  = 3.50cm]{geometry}

\usepackage[parfill]{parskip}
\pagenumbering{roman}
\title{A SOLUTION TO THE  POPULATION GROWTH IN UGANDA}
\author{Nsubuga Francis  16/U/19985/EVE  216021708\thanks{Lecturer: Dr. Ernest Mwebaze}}\newpage
\date{%
    Makerere University\\%
    Feb 24, 2018
}


\begin{document}
\begin{titlepage}
\maketitle
\end{titlepage}




\newpage
\pagenumbering{arabic} 
\section{\textbf{ Introduction}} 

Population refers to the number of people living in a particular region or country for this case Uganda. And on the other hand population growth refers to the rate at which the population of a given area rises or falls in a given time range. The population of Uganda is rising at a high rate. People are not using brth control methods to try and reduce on the high rates of growth. Population growth needs to be monitored, the more it rises the more pressure we incur on our resources. 

 


\subsection{\textbf{Background}}
Population is at a very high rate in Uganda. For the last population census carried out, they have shown that there is a high population growth in the country. 
\bigbreak


 For this reason all the citizens in the country need to be registered and all this information provided to the government inorder to know the areas and regions and the population each of them has.

\bigbreak
Its is for this reason therefore and more i came up with an online form to be filled in by citizens of Uganda both in rural and urban areas of the country.This will simply require the farmer to fill it on their smart form and data will be compared with what already exists
in the database.


\subsection{\textbf{Problem Statement}}

If a country properly knows the current population, it can easily budget for the resources. Resource usage needs to be monitored inorder to carter for the years to come.The high rates of population growth  can be controlled by educatig the citizens about the dangers of over population in agiven area/ country. Information about every individual needs to be captured and updately yearly to maintain/ know the exact population of the country.Citizens can personally register themselves online using the app.


\subsection{\textbf{Objectives}}


\subsubsection{\textbf{Main Objective}} 
The main objective of this report therefore is to know the number of children a given person has and properly monitor the birth rate and death rates in a given.

\subsubsection{\textbf{Specific Objectives}}

\begin{itemize}
  \item To collect all the data about a given person. 
  \item To perform a thorough analysis on the collected data about a particular individual.
  \item To come up with a conclusion from the data analysis about the population growth.
\end{itemize}


\subsection{\textbf{Scope}}
The research is mainly for all people  both in rural and urban areas.All the individuals in the country will have to be registered inorder to monitor the children being born and elderly dying in a particular year.

\subsection{\textbf{Research Significance}}
The aim of the research is to  solve the problem of approximations about the country's current population. If every child born is registered, then it will be easyto know the population grwoth annually. It also improves on the country's knowledeg about the different regions in the country that are over populated and need more resources so that they are properly cartered for in the budget. The country will also know the regions of the country that are sparsely populated so that they also have their chance at development. I believe this research is going to help our country know its population

\section{\textbf{Methodology}}
The proposed methodology consists of two phases, data collection and data analysis.\bigbreak
Data will be collected using ODK Collect, which will later on be uploaded to the ODK aggregate server to carry out all the required analysis.

\begin{itemize}
\item Enter Your Full Name: \textit{Nsubuga Francis}
\item Image: \textit{j.jpeg}
\item Enter Your Phone Number: \textit{0772497630}
\item District: \textit{Wakiso}
\item Village: \textit{Nsaggu}
\item No of children: \textit{4}
\item 1st child: \textit{Nsubuga Martin}
\item 2nd child: \textit{Nsubuga Matthew}
\item3rd child: \textit{Nsubuga Mark}
\item 4th child: \textit{Mubiru Ivan}
\item Date of Compiletion: \textit{25/02/2018}

\end{itemize}

 

\end{document}